\documentclass[a4paper, twocolumn]{article}

\usepackage[english]{babel}
\usepackage[utf8]{inputenc}
\usepackage{amsmath}
\usepackage{graphicx}
\usepackage{amssymb}
\usepackage{amsthm}
\usepackage{tikz-cd}
\usepackage{mathrsfs}
\usepackage[colorinlistoftodos]{todonotes}
\usepackage{enumitem}
%\usepackage{kpfonts}
%\usepackage{stackengine}
\usepackage{calc}
\usepackage[makeroom]{cancel}
\usepackage{marginnote}
\usepackage{verbatim}
\usepackage{listings}
\usepackage{mathtools}
%\usepackage{yfonts}
\setlength{\parindent}{0pt}
\setlength{\parskip}{0.4em}


\title{Elettrodinamica classica}
\author{Claudio Chiappetta}

\date{\today}
\newtheorem{thm}{Theorem}[section]
\newtheorem{lem}[thm]{Lemma}

\newtheorem{defn}[thm]{Definition}
\newtheorem{eg}[thm]{Example}
\newtheorem{ex}[thm]{Exercise}
\newtheorem{conj}[thm]{Conjecture}
\newtheorem{cor}[thm]{Corollary}
\newtheorem{claim}[thm]{Claim}
\newtheorem{rmk}[thm]{Remark}

\newcommand{\ie}{\emph{i.e.} }
\newcommand{\cf}{\emph{cf.} }
\newcommand{\into}{\hookrightarrow}
\newcommand{\dirac}{\slashed{\partial}}
\newcommand{\R}{\mathbb{R}}
\newcommand{\C}{\mathbb{C}}
\newcommand{\Z}{\mathbb{Z}}
\newcommand{\N}{\mathbb{N}}
\newcommand{\LieT}{\mathfrak{t}}
\newcommand{\T}{\mathbb{T}}
\newcommand\abs[1]{\left|#1\right|}
\newcommand{\angstrom}{\textup{\AA}}

\newlength\shlength
\newcommand\xshlongvec[2][0]{\setlength\shlength{#1pt}%
	\stackengine{-5.6pt}{$#2$}{\smash{$\kern\shlength%
			\stackengine{7.55pt}{$\mathchar"017E$}%
			{\rule{\widthof{$#2$}}{.57pt}\kern.4pt}{O}{r}{F}{F}{L}\kern-\shlength$}}%
	{O}{c}{F}{T}{S}}

\newcommand*\mean[1]{\overline{#1}}

\newcommand\pro{\item[$+$]}
\newcommand\con{\item[$-$]}

\newcount\colveccount
\newcommand*\colvec[1]{
        \global\colveccount#1
        \begin{pmatrix}
        \colvecnext
}
\def\colvecnext#1{
        #1
        \global\advance\colveccount-1
        \ifnum\colveccount>0
                \\
                \expandafter\colvecnext
        \else
                \end{pmatrix}
        \fi
}


\begin{document}
\maketitle




%\begin{abstract}
%Abstract a caso:

%\begin{itemize}
%	\item Presentazione del corso
 % \item Introduzione ai metodi di simulazione
  %\item Discussione del metodo di Eulero
  %\item 
%\end{itemize}
%\end{abstract}

%\marginpar[left text]{Lezione 1}

\section{Maxwell equations}
\subsection{idk lol}
\begin{align}
	\vec{A}(\mathbf{x},t)=\int d^3\mathbf{x'}dt'\vec{J}(\mathbf{x'},t) \frac{\delta (t-t'- \frac{\left|\mathbf{x'}-\mathbf{x}\right|}{c})}{\left|\mathbf{x'}-\mathbf{x}\right|} \tag{$\alpha$ costanti se S.I.} \\
		\phi (\mathbf{x},t)=\int d^3\mathbf{x'}dt' \rho(\mathbf{x'},t) \frac{\delta (t-t'- \frac{\left|\mathbf{x'}-\mathbf{x}\right|}{c})}{\left|\mathbf{x'}-\mathbf{x}\right|} \tag{$\alpha$ altre costanti se S.I.}
\end{align}
\subsection{Waves in dielectrics}
\begin{equation}
	\mathrm{D}=\epsilon_0 \mathrm{E} + \mathrm{P}
\end{equation}
\begin{equation}
\begin{cases}
\frac{\epsilon(\omega)}{\epsilon_0} =1 + \omega_p ^2 \sum_j \frac{\frac{f_j}{Z}}{w_j^2 - \gamma_j \omega - \omega^2} \\
\epsilon_p=\frac{Z e^2 N}{\epsilon_0 m}
\end{cases}
\end{equation}
\begin{equation}
	\omega^2=k^2c^2+\omega_p^2
\end{equation}
\begin{equation}
	\sigma_{Drude}=\frac{N f_0 e^2}{m \gamma_0 \epsilon_0}
\end{equation}
\begin{equation}
	\frac{\epsilon(\omega)}{\epsilon_0} = 1 -\frac{\omega_p^2}{\omega^2} ~~\text{Per i plasmi}
\end{equation}
\begin{equation}
	u(x,t)=\int_{-\infty}^{\infty}d\omega \left[\frac{2}{1+n(\omega)}\right]A(\omega) e^{i(k \cdot x-\omega t)} \tag{onda piana incidente in mezzo con $n(\omega)$}
\end{equation}

\section{Special relativity}

\subsection{Introduction}
\begin{equation}
  ds= \frac{d\tau}{\gamma}
\end{equation}

Trasformazioni delle velocità, dove $\mathbf{u}$ è la velocità di traslazione fra i due sistemi, e $\mathbf{v}$ è la velocità della particella nel primo sistema
\begin{gather}
  \begin{split}
    v_{\parallel}=\frac{v'_{\parallel}+u}{1+\frac{\mathbf{v'} \cdot \mathbf{u}}{c^2}} \\
    \mathbf{v}_{\bot}=\frac{\mathbf{v}'_{\bot}}{\gamma (1+\frac{\mathbf{v'} \cdot \mathbf{u}}{c^2})} \\
  \end{split}\\
  \begin{aligned}[t]
    v'_{\parallel}=\frac{v_{\parallel}-u}{1-\frac{\mathbf{v'} \cdot \mathbf{u}}{c^2}}     
  \end{aligned}
\end{gather}

Supponendo $\mathbf{u}=u\mathbf{\hat{x}}$
\begin{equation}
  a_{x}=
\end{equation}

e:
\begin{equation}
  a_{\bot}=\frac{a'_\bot+\left[ \right]}{denominator}
\end{equation}

[Lasciamo perdere!]

Questo quadrivettore velocità è invariante
\begin{equation}
  u^\mu \coloneqq \frac{dx^\mu}{d\tau}=\begin{pmatrix}c \gamma\\ \mathbf{v}\gamma\end{pmatrix}%=\colvec{2}{a}{b}
\end{equation}
Vediamo ora il quadrivettore accelerazione:
\begin{align}
  a^\mu \coloneqq \frac{d u^\mu}{d\tau}=\gamma \colvec{2}{c \frac{d\gamma}{d t}}{\frac{d\gamma}{d t}\mathbf{v}+\gamma\mathbf{a}}=\colvec{2}{c \gamma^4\dot{\beta}\cdot \beta}{\gamma^4\dot{\beta}\cdot{\beta}\mathbf{v}+\gamma\mathbf{a}} \\
  a^2=-\gamma ^6 \left[ a^2-\frac{(\mathbf{v}\times \mathbf{a })^2}{c^2} \right]
\end{align}

\begin{align}
  \label{eq:2}
  \mathcal{L}=-\frac{mc^2}{\gamma} \\
  \mathbf{p}=\frac{d\mathcal{L}}{d\mathbf{v}}=m\gamma\mathbf{v} \\
  H = \mathbf{p}\cdot\mathbf{v}-\mathcal{L} \tag{Hamiltonian}=m\gamma c^2 =\epsilon
\end{align}
Introduciamo il quadrivettore momento:
\begin{align}
  \label{eq:3}
  p^\mu=mv^\mu=m \colvec{2}{\gamma c}{\gamma \mathbf{u}}=\colvec{2}{\frac{\epsilon}{c}}{\mathbf{p}} \\
  p^2=mc^2
\end{align}

Consideriamo ora un'onda piana, abbiamo \textbf{invarianza della fase}, poichè la fase è un conteggio di creste
\begin{align}
  \label{eq:4}
  \phi=k\cdot\mathbf{x}-\omega t=k'\cdot\mathbf{x'}-\omega' t'
\end{align}
Da qui, sostituendo $x'^\mu$ usando il boost di Lorentz, ricavo l'ultimo quadrivettore:
\begin{align}
  \label{eq:5}
  k^\mu=\colvec{2}{\frac{\omega}{c}}{\mathbf{k}}
\end{align}
Queste formule contengono l'effetto Doppler e la legge di aberrazione:
\begin{align}
  \label{eq:6}
  \omega'=\gamma\omega(1-\beta\cos\theta)
  \tan \theta'=\frac{\sin\theta}{\gamma\cos\theta-\beta}
\end{align}
\begin{align}
  \label{eq:7}
  \frac{dp^\mu}{d\tau}=F^\mu
\end{align}

\subsection{Covarianza dell'elettrodinamica}
\label{sec:covar-dell}
\begin{align}
  \label{eq:8}
  \frac{d}{d\tau} \colvec{2}{p_0}{\mathbf{p}}=\colvec{2}{\frac{q}{c} \mathbf{u}\cdot\mathbf{E}}{\frac{q}{c}( u_0\mathbf{E}+\mathbf{u}\times\mathbf{B})} %\\
%  con~u^\mu \coloneqq \colvec{2}{c\gamma}{\mathbf{v}\gamma}
\end{align}
Voglio che il membro di destra sia un quadrivett, per cui introduco:
\begin{align}
  \label{eq:1}
  J^\mu\coloneqq \colvec{2}{\rho c}{\rho \frac{dx}{dt}} \\
  \partial^\mu J_\mu=\frac{\partial \rho}{\partial t} + \nabla\cdot\mathbf{J}
\end{align}
\begin{align}
  \partial^\mu A_\mu \tag*{gauge di Lorenz} \nonumber \\
  \Box A^\mu=4\pi J^\mu  \label{eq:9}
\end{align}
Da cui:
\begin{align}
  \label{eq:10}
  F^{\mu\nu}\coloneqq \partial^\mu A^\nu - \partial^\nu A^\mu \\
  F^{\mu\nu}=
  \begin{pmatrix}
    0 & -E_x & -E_y & -E_z \\
    E_x & 0 & -B_z & B_y \\
    E_y & B_z & 0 & -B_x \\
    E_z & -B_y & B_x & 0 \\
  \end{pmatrix} \\
  F^{\mu\nu}=(\mathbf{E}, \mathbf{B}) \\
  F_{\mu \nu}=(-\mathbf{E}, \mathbf{B}) \\
  F^{*\mu \nu}=(\mathbf{B}, \mathbf{E}) \\
  F^*_{\mu \nu}=(-\mathbf{B}, -\mathbf{E})
\end{align}
Riscriviamo le eq. di Maxwell
\begin{align}
  \label{eq:12}
  \partial_\mu F^{\mu \nu}=\frac{4\pi}{c}J^\nu \\
  \partial_\mu F^{*\mu \nu}=0 \\
  \partial^\mu F^{\nu\rho}+\partial^\rho F^{\mu\nu}+\partial^\nu F^{\rho\mu}=0 \tag{forma alternativa per la seconda}
\end{align}
Posso riscrivere le eq. del moto in forma covariante
\begin{align}
  \label{eq:13}
  \frac{dp^\mu}{d\tau}=m\frac{du^\mu}{d\tau}=\frac{q}{c}F^{\mu\nu}u_\nu
\end{align}

\subsection{Leggi di trasformazione dei campi}
\label{sec:leggi-di-trasf}

\begin{align}
  \label{eq:14}
  stranote
\end{align}
Vediamo alcuni invarianti
\begin{align}
  \label{eq:15}
  \mathbf{E}^2-\mathbf{B}^2=cost \\
  \mathbf{E}\cdot\mathbf{B}=cost
\end{align}

\subsection{Lagrangiana e Hamiltoniana di particella}
\label{sec:lagr-e-hamilt}
Un po' di formule a caso
\begin{align}
  \label{eq:16}
  \mathcal{L}_{free}=-\frac{mc^2}{\gamma} \\
  \mathcal{L}\gamma=cost \\
  \frac{d}{dt}\frac{d\mathcal{L}}{d\mathbf{v}}=\frac{d\mathcal{L}}{d\mathbf{x}} \\
  \frac{d\mathcal{L}_{free}}{d\mathbf{x}}=0 \\
\end{align}

\subsection{Soluzione all'eq. delle onde in forma covariante}
\label{sec:soluz-alleq-delle}
Risolviamo l'equazione \ref{eq:9} a pagina \pageref{eq:9}, supponendo $J^\mu=J^\mu(x)$, utilizzando una funzione di Green:
\begin{align}
  \label{eq:18}
  \Box_x D(x-x')=\delta^{(4)}(x-x') \\
  z\coloneqq x-x'
\end{align}
Passando ad uno spazio di Fourier si ha
\begin{align}
  \label{eq:19}
  D(k)=\frac{1}{k\cdot k} \\
  D(z)=-\frac{1}{(2\pi)^4}\int dk D(k) e^{-ik\cdot x}
\end{align}
Risolvendo, si hanno due soluzioni:
\begin{align}
  \label{eq:20}
  D_{ritardata}=\frac{1}{2\pi}\theta(x_0-x'_0)\delta[(x-x')^2] \\
  D_{anticipata}=\frac{1}{2\pi}\theta(x'_0-x_0)\delta[(x-x')^2] 
\end{align}

\section{Moving charges}
\label{sec:mov-charges}

Posso scrivere il quadrivettore delle sorgenti per una carica in moto come:
\begin{align}
  \label{eq:21}
  J^\mu=qc \int d\tau u^\mu(\tau)\delta^{(4)}(x-r(\tau)) \\
  u^\mu\coloneqq \colvec {2}{\gamma c}{\gamma \mathbf{v}} \qquad r(t)\coloneqq \colvec {2}{ct}{r(t)}
\end{align}

\subsection{Lienerd-Wichert}
\label{sec:lienerd-wichert}
Partiamo trovando i potenziali
\begin{align}
  \label{eq:17}
  A^\mu(x)=\frac{4\pi}{c}\int d^4x'D_r(x-x')J^\mu(x')
\end{align}
Sostituendo l'eq \ref{eq:21} a pagina \pageref{eq:21}, si ottiene



\section{Notazione}
\label{sec:notazione}
\begin{align}
  \label{eq:11}
  X_{\mu} \tag*{covariante} \\
  X^{\mu} \tag*{controvariante} \\
\end{align}

\section{M.U.}
\begin{equation}
	1 ~\mathrm{eV} \approx 1.6 \cdot 10^{-19} \mathrm{J}
\end{equation}
\end{document}
%%% Local Variables:
%%% mode: latex
%%% TeX-master: t
%%% End:

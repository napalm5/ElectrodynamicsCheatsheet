\documentclass[a4paper, twocolumn]{article}

\usepackage[english]{babel}
\usepackage[utf8]{inputenc}
\usepackage{amsmath}
\usepackage{graphicx}
\usepackage{amssymb}
\usepackage{amsthm}
\usepackage[inter-unit-product =\cdot]{siunitx}
\usepackage{tikz-cd}
\usepackage{mathrsfs}
\usepackage[colorinlistoftodos]{todonotes}
\usepackage{enumitem}
\usepackage{url}
% \usepackage{kpfonts}
%\usepackage{stackengine}
\usepackage{calc}
\usepackage[makeroom]{cancel}
\usepackage{marginnote}
\usepackage{verbatim}
\usepackage{listings}
\usepackage{mathtools}
%\usepackage{yfonts}
\setlength{\parindent}{0pt}
\setlength{\parskip}{0.4em}


\title{Classical Electrodynamics}

\date{\today}

%some definitions, courtesy of Gabriele Bozzola (https://github.com/Sbozzolo)
\newtheorem{thm}{Theorem}[section]
\newtheorem{lem}[thm]{Lemma}

\newtheorem{defn}[thm]{Definition}
\newtheorem{eg}[thm]{Example}
\newtheorem{ex}[thm]{Exercise}
\newtheorem{conj}[thm]{Conjecture}
\newtheorem{cor}[thm]{Corollary}
\newtheorem{claim}[thm]{Claim}
\newtheorem{rmk}[thm]{Remark}

\newcommand{\ie}{\emph{i.e.} }
\newcommand{\cf}{\emph{cf.} }
\newcommand{\into}{\hookrightarrow}
\newcommand{\dirac}{\slashed{\partial}}
\newcommand{\R}{\mathbb{R}}
\newcommand{\C}{\mathbb{C}}
\newcommand{\Z}{\mathbb{Z}}
\newcommand{\N}{\mathbb{N}}
\newcommand{\LieT}{\mathfrak{t}}
\newcommand{\T}{\mathbb{T}}
\newcommand\abs[1]{\left|#1\right|}
\newcommand{\angstrom}{\textup{\AA}}

\newlength\shlength
\newcommand\xshlongvec[2][0]{\setlength\shlength{#1pt}%
	\stackengine{-5.6pt}{$#2$}{\smash{$\kern\shlength%
			\stackengine{7.55pt}{$\mathchar"017E$}%
			{\rule{\widthof{$#2$}}{.57pt}\kern.4pt}{O}{r}{F}{F}{L}\kern-\shlength$}}%
	{O}{c}{F}{T}{S}}

\newcommand*\mean[1]{\overline{#1}}

\newcommand\pro{\item[$+$]}
\newcommand\con{\item[$-$]}


%This is mine, instead
\newcount\colveccount
\newcommand*\colvec[1]{
        \global\colveccount#1
        \begin{pmatrix}
        \colvecnext
}
\def\colvecnext#1{
        #1
        \global\advance\colveccount-1
        \ifnum\colveccount>0
                \\
                \expandafter\colvecnext
        \else
                \end{pmatrix}
        \fi
}

%\sisetup{inter-unit-product =\cdot}
\DeclareSIUnit\dyne{dyne}
\DeclareSIUnit\gauss{G}

% to have multiple equations under the same tag
\newenvironment{eqs}
    {\begin{gather}
        \begin{split}
        }
        { 
        \end{split} 
      \end{gather}
    }

\newcommand\id{\textrm{d}}
    
    % cases to the right
%\newenvironment{rcases}
%{}
%{} 


\begin{document}
\maketitle
%\begin{abstract}
%Abstract a caso:

%\begin{itemize}
%	\item Presentazione del corso
 % \item Introduzione ai metodi di simulazione
  %\item Discussione del metodo di Eulero
  %\item 
%\end{itemize}
%\end{abstract}

%\marginpar[left text]{Lezione 1}

\section{Maxwell equations}
Let's start from the basics:

\begin{equation}
  \label{eq:33}
\begin{aligned}[c]
  \nabla\cdot\mathbf{D}=\rho \\
  \nabla\cdot\mathbf{B}=0 
\end{aligned}
\qquad
\begin{aligned}[c]
  \nabla\times\mathbf{E}+\frac{\partial\mathbf{B}}{\partial t}=0 \\
  \nabla\times\mathbf{H}-\frac{\partial\mathbf{D}}{\partial t}=\mathbf{J}
\end{aligned}
\end{equation}

\begin{gather}
  \label{eq:35}
  \begin{split}
    \mu_0=\SI{4\pi e-1}{\newton\per\ampere\squared} \\
    \epsilon_0=\SI{8.854e-12}{\farad\per\meter} \\
    c=\SI{2.998}{\meter\per\second} \\
    v=\frac{c}{n}
  \end{split}
\end{gather}

\begin{gather}
  \label{eq:36}
  \begin{split}
    D_i=D_i(\mathbf{E})=\sum_{j}\epsilon_{ij}\mathbf{E}_j+\mathrm{O}(E^2) \\
    H_i=H_i(\mathbf{B})=\sum_{j}\mu_{ij}\mathbf{B}_j+\mathrm{O}(B^2)
  \end{split}
\end{gather}

\begin{gather}
  \label{eq:59}
  \begin{split}
    W=\frac{1}{2}(\epsilon E^2+\mu B^2)
  \end{split}
\end{gather}

\subsection{Useful properties}
\label{sec:useful-properties}
\begin{gather}
  \label{eq:37}
  \begin{split}
    A\cdot(B\times C)=(A\times B)\cdot C \\
    A\times(B\times C)=B(A\cdot C)-C(A\cdot B)
  \end{split}
\end{gather}

Every field $\mathbf{A}$ can be decomposed in this way
\begin{align*}
  \begin{aligned}
    \mathbf{A}\coloneqq\mathbf{A}_l+\mathbf{A}_t \\
  \end{aligned}
  \qquad \text{\textbf{such that}}\qquad 
  \begin{aligned}
    \nabla\times\mathbf{A}_l=0 \\
    \nabla\cdot\mathbf{A}_t =0\\
  \end{aligned}
\end{align*}

\subsection{Useful theorems}
\label{sec:useful-theorems}

\begin{gather}
  \label{eq:38}
  \begin{split}
    \int \displaylimits_V d^3\textbf{x}~ \nabla\cdot\mathbf{A}=\int\displaylimits_S d\mathbf{s}\cdot \mathbf{A} \qquad   \text{\textbf{Gauss}} \\
    \int\displaylimits_S d\mathbf{s}\cdot(\nabla\times \mathbf{A})=\oint\displaylimits_C d\mathbf{l}\cdot\mathbf{A} \qquad \text{\textbf{Stokes}}
  \end{split} 
\end{gather}

\subsection{Maxwell equations}
\label{sec:maxwell-equations}
Using the two homogenous Maxwell equations, we define the potentials $\mathbf{A}$ and $\phi$; using the two inhomogenous, selecting the Lorenz gauge ($\nabla\cdot\mathbf{A}+\frac{1}{c^2}\frac{\partial}{\partial t}\phi=0$), we kinda obtain \emph{wave equations} (if in the vacuum, we obtain proper wave equations)
\begin{gather}
  \begin{aligned}[l]
      \label{eq:40}
      \nabla^2\phi -\frac{1}{c^2}\frac{\partial^2\phi}{\partial t^2}=-\frac{\rho}{\epsilon_0} \\
      \nabla^2\mathbf{A}-\frac{1}{c^2}\frac{\partial^2\mathbf{A}}{\partial t^2}= -\mu_0 \mathbf{J} 
  \end{aligned}
\end{gather}
With the Coulomb gauge, we obtain the Poisson equation.
We can solve eq. \ref{eq:40} by the means of the \emph{Green function} $\mathrm{G}(x,x',t,t')$
\begin{equation}
  \label{eq:34}
  (\nabla^2-\frac{1}{c^2}\frac{\partial^2}{\partial t^2})~\mathrm{G}(x,x',t,t')=-4\pi\delta(\mathbf{x}-\mathbf{x'})\delta(t-t')
\end{equation}
It is easy to find the fourier transform of $\mathrm{G}(x,x',t,t')$
\begin{gather}
  \label{eq:41}
  g(\mathbf{k},\omega)=\frac{1}{k^2-\frac{\omega^2}{c^2}} \\
  \mathrm{G}(x,x',t,t')=\int d^3\mathbf{k}d\omega g(\mathbf{k},\omega)e^{\mathbf{k}\cdot(\mathbf{x}-\mathbf{x'}) - \omega (t-t')}
\end{gather}
After some calculations (omitted), this is the retarded Green function
\begin{align}
  \label{eq:42}
  \mathrm{G}(\mathbf{x}-\mathbf{x'},t-t')=\mathrm{G}(\mathbf{R},\tau)=\frac{t-t'-\frac{\left|\mathbf{x}-\mathbf{x'} \right|}{c}}{\left|\mathbf{x}-\mathbf{x'} \right|}
\end{align}
And thus the potentials and the fields
\begin{gather}
  \begin{aligned}
    \label{eq:43}
    \Phi(\mathbf{x},t)=\frac{1}{4\pi\epsilon_0}\int d^3\mathbf{x'}\frac{\rho (\mathbf{x'}, t-\frac{\left|\mathbf{x}-\mathbf{x'} \right|}{c})}{\frac{\left|\mathbf{x}-\mathbf{x'} \right|}{c}} \\
    \mathbf{A}(\mathbf{x},t)=\frac{\mu_0}{4\pi}\int d^3\mathbf{x'}\frac{J (\mathbf{x'}, t-\frac{\left|\mathbf{x}-\mathbf{x'} \right|}{c})}{\frac{\left|\mathbf{x}-\mathbf{x'} \right|}{c}} \\
  \end{aligned}\\
  \begin{aligned}
    \label{eq:44}
    \mathbf{E}(\mathbf{x},t)=\frac{1}{4\pi\epsilon_0}\int d^3\mathbf{x'}\frac{1}{R}\left[-\nabla'\rho-\frac{1}{c^2}\frac{\partial}{\partial t'}\mathbf{J}  \right]_{rit} \\
    \mathbf{B}(\mathbf{x},t)=\frac{\mu_0}{4\pi}\int d^3\mathbf{x'}\frac{1}{R}\left[\nabla'\times \mathbf{J} \right] \qquad \qquad \\
  \end{aligned}
\end{gather}
We can separate eq. \ref{eq:44} into a static and a time dependent term, obtaining the \emph{Jefimenko equations} (omitted)

\subsection{Continuity equation}
\label{sec:continuity-equation}
In general, when a quantity $f$ is conserved, the \emph{continuity equation yields}:
\begin{align}
  \label{eq:44}
  \frac{\partial \rho_f}{\partial t}+\nabla\cdot\mathbf{J}_f=\left(\frac{\partial \rho_f}{\partial t}\right)_S
\end{align}
From Maxwell equation, we obtain the \emph{Poynting theorem} (conservation of energy)
\begin{align}
  \label{eq:45}
  \frac{\partial u}{\partial t}+\nabla\cdot S=-\mathbf{J}\cdot\mathbf{E}
\end{align}
We can do the same thing for the conservation of the moment.

%\subsection{idk lol}
%\begin{align}
%	\vec{A}(\mathbf{x},t)=\int d^3\mathbf{x'}dt'\vec{J}(\mathbf{x'},t) \frac{\delta (t-t'- \frac{\left|\mathbf{x'}-\mathbf{x}\right|}{c})}{\left|\mathbf{x'}-\mathbf{x}\right|} \tag{$\alpha$ costanti se S.I.} \\
%		\phi (\mathbf{x},t)=\int d^3\mathbf{x'}dt' \rho(\mathbf{x'},t) \frac{\delta (t-t'- \frac{\left|\mathbf{x'}-\mathbf{x}\right|}{c})}{\left|\mathbf{x'}-\mathbf{x}\right|} \tag{$\alpha$ altre costanti se S.I.}
   %  \end{align}

\subsection{Plane waves and wave propagation}
\label{sec:plane-waves-wave}

Considering a plane wave propagating in the $z$ direction, or $\mathbf{E}=\mathbf{E}(z,t)$. Then the Maxwell equations can be written as
\begin{align}
  \label{eq:46}
  omitted
\end{align}
blabla
\begin{gather}
  \label{eq:49}
  \begin{aligned}
    Z=\sqrt{\frac{\mu}{\epsilon}} \\
    v=\frac{1}{\epsilon Z}=\frac{Z}{\mu}=\frac{1}{\sqrt{\epsilon\mu}} \\
    n=\sqrt{\frac{\epsilon}{\epsilon_0}\frac{\mu}{\mu_0}}
  \end{aligned}
\end{gather}
One more time, from this set of equations, we can obtain \emph{wave equations}. (this part may be omitted)
\begin{gather}
  \label{eq:50}
  \begin{aligned}
    \frac{\partial \mathbf{E}}{\partial z}=-\frac{1}{v}\frac{\partial}{\partial t}(Z\mathbf{H}\times\hat{z}) \\
    \frac{\partial (Z\mathbf{H}\times\hat{z})}{\partial z}=-\frac{1}{v}  \frac{\partial \mathbf{E}}{\partial t} 
  \end{aligned}
\end{gather}
Which has solution (with $\mathbf{F}$,$\mathbf{G}$ arbitrary functions)
\begin{gather}
  \label{eq:52}
  \begin{aligned}
    \mathbf{E}(z,t)=\mathbf{F}(z-vt)+\mathbf{G}(z+vt) \\
    \mathbf{B}(z,t)=\frac{1}{Z}\hat{z}\left[\mathbf{F}(z-vt)-\mathbf{G}(z+vt)\right]
  \end{aligned}
\end{gather}
Let's now see what happes in a lossy medium. We have
\begin{gather}
  \label{eq:53}
  \begin{split}
    \mathbf{J}_{cond}=\sigma\mathbf{E} \qquad \\
    \mathbf{J}_{disp}=-i \omega\mathbf{D}=-i\omega\epsilon_d\mathbf{E}  \\
    \mathbf{J}=\mathbf{J}_{cond}+\mathbf{J}_{disp}=-i\omega\epsilon_c\mathbf{E}
  \end{split}
\end{gather}
With the same logic as before, but this time with different starting equation, we still obtain wave equations for the fields
\begin{gather}
  \label{eq:55}
  \begin{aligned}
    \mathbf{E}=\mathbf{E}_{0+}e^{ik_cz}+\mathbf{E}_{0-}e^{-ik_cz} \\
    \mathbf{H}=\frac{1}{z_c}\hat{k}\times\mathbf{E} \qquad
  \end{aligned}\\
  \begin{aligned}
    k_c=\beta+i \frac{\alpha}{2}=\omega\sqrt{\mu\epsilon_d}(1+i\tau)^{\frac{1}{2}}=\frac{\omega\mu}{Z_c} \\
    \tau\coloneqq\frac{\mathrm{Im}(\epsilon_c)}{\mathrm{Re}(\epsilon_c)} =\frac{\frac{\sigma_s}{\omega}+\mathrm{Im}(\epsilon_d)}{\mathrm{Re}(\epsilon_d)}
  \end{aligned}
\end{gather}
Where $\beta$ is the \emph{wave number} and $\alpha$ is the \emph{absorption coefficient}. The following relations hold:
\begin{gather}
  \label{eq:57}
  \begin{split}
    \alpha=\beta \frac{\mathrm{Im}(\frac{\epsilon}{\epsilon_o})}{\mathrm{Re}\frac{\epsilon}{\epsilon_0}} \\
    \beta=k_0\sqrt{\mathrm{Re}\frac{\epsilon}{\epsilon_0}}
  \end{split}
\end{gather}
\begin{gather}
  \label{eq:58}
  \begin{split}
    k=\omega\sqrt{\mu\epsilon}
  \end{split}
\end{gather}

If we take a wave propagatin in a generic direction $\hat{x}$, we can reduce it to the precedent (propagating along $\hat{z}$) case by the means of a rotation. Then the wave can be \textbf{TE} ($E\bot(x,y) $)or \textbf{TM} ($H\bot(x,y)$) [pg. 27]

We have planes of constant amplitude ($\mathbf{\alpha}\cdot\mathbf{x}=\text{const}$), and planes of constant phase ($\mathbf{\beta}\cdot\mathbf{x}=\text{const}$); in a uniform wave they are the same ($\hat{\alpha}=\hat{\beta}=\hat{k}$).

If a medium has $\epsilon<0$ and $\mu<0$, then $k<0$, and we have the following effects:
\begin{itemize}
\item Negative Doppler effect
\item Inverted Cerenkov effect
\item Inverted diffraction (the diffraction angle is inverted)
\end{itemize}
The ... has the following expression
\begin{equation}
  \label{eq:60}
  W=\frac{1}{2}\left[\frac{\partial}{\partial\omega}(\epsilon\omega)E^2+\frac{\partial}{\partial\omega}(\mu\omega)B^2\right]
\end{equation}
$\implies$ where $\epsilon\mu<0$ the wave can't propagate (consequences about refraction and reflection).

Let's now see what happens at the interface between two media, considering an interface at $z=0$, and a plane wave normally incident; we can study \emph{dynamic properties} and \emph{cinematic properties}.

Cinematic properties can be determined from the boundary conditions.
\begin{gather}
  \label{eq:39}
  \begin{split}
    \omega^i=&~\omega^t=\omega^r\\
    \mathbf{k}_i\cdot\mathbf{x}\vert_{z=0}=&~\mathbf{k}_t\cdot\mathbf{x}\vert_{z=0}=\mathbf{k}_r\cdot\mathbf{x}\vert_{z=0} \\
    \frac{\sin i}{\sin t}=&~\frac{k_t}{k}=\frac{n'}{n} 
  \end{split}
\end{gather}
For the dynamic properties, we recall the following laws, derived from the Maxwell equations
\begin{gather}
  \label{eq:47}
  \begin{split}
    (\mathbf{D}_2-\mathbf{D}_1)\cdot\hat{n}=\sigma\\
    (\mathbf{B}_2-\mathbf{B}_1)\cdot\hat{n}=0\\
    \hat{n}\times(\mathbf{E}_2-\mathbf{E}_1)=0\\
    \hat{n}\times(\mathbf{H}_2-\mathbf{H}_1)=\mathbf{J}
  \end{split}  
\end{gather}
Considering TE ($\mathbf{E}=E\hat{y}$), we find the \emph{Fresnel formulas}


\subsection{Frequency Dispersion Characteristics of Dielectrics,
Conductors, and Plasmas}
\begin{equation}
  \begin{split}
    \mathrm{D}=&~\epsilon_0 \mathrm{E} + \mathrm{P} %\\
    %\implies \frac{\epsilon(\omega)}{\epsilon_0}=1+\frac{Ze^2N}{\epsilon_0m}&~\sum_j \frac{f_j Z}{\omega_j^2-i\gamma\omega-\omega^2}
  \end{split}
\end{equation}
\begin{gather}
  \begin{cases}
    \frac{\epsilon(\omega)}{\epsilon_0} =1 + \omega_p ^2 \sum_j \frac{\frac{f_j}{Z}}{w_j^2 - \gamma_j \omega - \omega^2} \\
    \epsilon_p=\frac{Z e^2 N}{\epsilon_0 m}
  \end{cases}\\
  \begin{split}
    \omega_j\approx \SI{e15}{\hertz};~ \gamma_j\approx 10^{11\div 12}\si{\hertz};~ w_p\approx \SI{e16}{\hertz}\\
    \text{$\omega_j>>\gamma_j~\implies~\epsilon(\omega)~$ in generale è reale}
  \end{split}\\  
  \begin{aligned}[t]
    \frac{\epsilon(\omega)}{\epsilon_0} =&~1 + \omega_p ^2\sum_j \frac{\frac{f_j}{Z}(\omega_j^2-\omega^2)}{(\omega_y^2-\omega^2)^2+\gamma^2_j\omega^2}+\\
    &~i\omega_p^2\sum_j \frac{\frac{f_j}{Z}\gamma_j\omega}{(\omega_y^2-\omega^2)^2+\gamma^2_j\omega^2}
  \end{aligned}
\end{gather}
If the charateristic frequancy $\omega_0=0$, we are modeling a conductor (there are free electrons).
\begin{align}
  \label{eq:48}
  \begin{aligned}[t]
    \frac{\epsilon(\omega)}{\epsilon_0}=&~1-\frac{\omega_p^2 \frac{f_0}{Z}}{\omega(\omega+i\gamma_0)}+\sum_{j\neq0}\frac{\omega_p^2 \frac{f_j}{Z}}{\omega_j^2-i\gamma_j\omega-\omega^2} \\
    &\xrightarrow[\omega\to0]{}\epsilon_{bound}+i \frac{\omega_p^2 \frac{f_0}{Z}}{\omega(\gamma_0-i\omega)}
\end{aligned}
\end{align}

\begin{equation}
	\omega^2=k^2c^2+\omega_p^2
\end{equation}
Using the aproximation of small $\omega$, we can develop the \emph{Drude model}
\begin{equation}
	\sigma_{Drude}=\frac{N f_0 e^2}{m \gamma_0 \epsilon_0}
\end{equation}
If $\sigma_{Drude}$ is small, but not too small, it also has a complex component

\subsection{Ionosphere propagation}
\label{sec:ionosph-prop}
\begin{gather}
  \label{eq:51}
  \begin{rcases}
    \omega_j=0  & \forall~~ j \\
    \gamma_0=0 & 
  \end{rcases}
  \begin{aligned}[c]
    \qquad \epsilon(\omega)=\left(1-\frac{\omega_p^2}{\omega^2}  \right)\epsilon_o
  \end{aligned}
\end{gather}
In this case, the magnetic field is not negligible ($B\sim0.1\div 1\si{\gauss}$).
Considering, $B$ in the $\hat{z}$ direction, and a circularly polarized radiation propagating along the direction $\hat{z}$, we find
\begin{align}
  \label{eq:54}
  \begin{split}
    x_0=\frac{e\left(\frac{E_o}{n}\right)}{\omega(\omega\mp\omega_B)} \\
    \frac{\epsilon(\omega)}{\epsilon_0}=1-\frac{\omega_p^2}{\omega(\omega\mp\omega_B)}
  \end{split}
\end{align}
So, if we launch a radiation from earth to the ionosphere, we see that, separating the polarization into a clockwise ad a counterclockwise component, one of the two can propagate, while the other one cannot. this is an example of \emph{birefrengence}.
%\begin{equation}
%	\frac{\epsilon(\omega)}{\epsilon_0} = 1 -\frac{\omega_p^2}{\omega^2} ~~\text{Per i plasmi}
% \end{equation}

\subsection{Group velocity}
\label{sec:group-velocity}
What happens to a wave packet in a medium? Let's consider a multimode wave
\begin{equation}
  \label{eq:56}
  E(z,t)=\int\limits_{-\infty}^{\infty}\mathrm{d}k A(k)e^{ikz-i\omega t}
\end{equation}
A typical packet has different $k$, but centered around a $k_0$ and with a small spread. so we can expand, and by ignoring the orders $\geq2$ we can write
\begin{equation}
  \label{eq:63}
  E(z,t)=e^{i(V_gk_0-\omega_0t)}\int\mathrm{d}k A(k)e^{ik(z-V_gt)}
\end{equation}
So we have a packet that is rigidly translated at a velocity $v_g~\implies~$ energy is rigidly transferred at grup velocity.

We can also find $v_g$ by noting that $\frac{\mathrm{d}k}{\mathrm{d}k}=\frac{\mathrm{d}\omega n(\omega)c^{-1}}{\mathrm{d}k}$
In conclusion:
\begin{align}
  \label{eq:64}
  \begin{split}
    f_g=\frac{1}{n+\omega \frac{\mathrm{d}n}{\mathrm{d}\omega}}=\frac{\mathrm{d}\omega}{\mathrm{d}k}
    v_f=\frac{c}{n}=\frac{\omega}{k}
  \end{split}
\end{align}
Considering also higher orders, we find that $v_g$ can be higher than $c$, and also negative. So how do we interpret $v_g$y? It's not the speed of the information transfer; we transfer information with a discontinuity in the progation, so information always travels at speed $c$. $v_g$ looses meaning in regions with anomal dispersion. E.g. when there is absorption, the packet is highly distorted. We have to be careful when the approximation no longer yields


\subsection{Arrival of a signal after propagation through
a dispersive medium}
\label{sec:arrival-signal-after}
\begin{align}
  \begin{split}
    u(z,t)=\int_{-\infty}^{\infty}\mathrm{d}\omega \left[\frac{2}{1+n(\omega)}\right]A(\omega) e^{i(\frac{\omega n(\omega)}{c}z-\omega t)} 
  \end{split}
\end{align}
(not really important)

\subsection{Causality in the connection between D and E}
\label{sec:caus-conn-betw}

We saw that
\begin{equation}
  \label{eq:66}
  \mathbf{D}(\mathbf{x},\omega)=\epsilon(\omega)\mathbf{E}(\mathbf{x},\omega)
\end{equation}
With some Fourier transform we can find
\begin{align}
  \label{eq:67}
  \begin{split}
    \mathbf{D}(\mathbf{x},t)=&~\epsilon_0\mathbf{E}(\mathbf{x},t)+\epsilon_0\int\mathrm{d}\tau G(\tau)\mathbf{E}(\mathbf{x},t-\tau) \\
    G(\tau)&~\coloneqq \frac{1}{2\pi}\int\id \omega\left[\frac{\epsilon(\omega)}{\epsilon_0}-1\right]e^{-i\omega\tau} \\
    \tau&~\coloneqq t-t'
  \end{split}
\end{align}
This equation is not local in time. A couple observations:
\begin{itemize}
\item If the medium is not dispersive ($\epsilon(\omega)=const$), then the equation is local in time
\item By calculation, we can find that $G(\tau)=0$ if $\tau<0$ (causality is conserved).
\end{itemize}
So far we didn't consider a spatial non-locality; in that, more general, case, we would have $\epsilon=\epsilon(k,w)$

...

There is a relation between the real and the imaginary part of $\epsilon$. Measuring one, the other can be obtained


\section{Special relativity}

\subsection{Introduction}
\begin{equation}
  ds= \frac{d\tau}{\gamma}
\end{equation}

Trasformazioni delle velocità, dove $\mathbf{u}$ è la velocità di traslazione fra i due sistemi, e $\mathbf{v}$ è la velocità della particella nel primo sistema
\begin{gather}
  \begin{split}
    v_{\parallel}=\frac{v'_{\parallel}+u}{1+\frac{\mathbf{v'} \cdot \mathbf{u}}{c^2}} \\
    \mathbf{v}_{\bot}=\frac{\mathbf{v}'_{\bot}}{\gamma (1+\frac{\mathbf{v'} \cdot \mathbf{u}}{c^2})} \\
  \end{split}\\
  \begin{aligned}[t]
    v'_{\parallel}=\frac{v_{\parallel}-u}{1-\frac{\mathbf{v'} \cdot \mathbf{u}}{c^2}}     
  \end{aligned}
\end{gather}

Supponendo $\mathbf{u}=u\mathbf{\hat{x}}$
\begin{equation}
  a_{x}=
\end{equation}

e:
\begin{equation}
  a_{\bot}=\frac{a'_\bot+\left[ \right]}{denominator}
\end{equation}

[Lasciamo perdere!]

Questo quadrivettore velocità è invariante
\begin{equation}
  u^\mu \coloneqq \frac{dx^\mu}{d\tau}=\begin{pmatrix}c \gamma\\ \mathbf{v}\gamma\end{pmatrix}%=\colvec{2}{a}{b}
\end{equation}
Vediamo ora il quadrivettore accelerazione:
\begin{gather}
  a^\mu \coloneqq \frac{d u^\mu}{d\tau}=\gamma \colvec{2}{c \frac{d\gamma}{d t}}{\frac{d\gamma}{d t}\mathbf{v}+\gamma\mathbf{a}}=\colvec{2}{c \gamma^4\dot{\beta}\cdot \beta}{\gamma^4\dot{\beta}\cdot{\beta}\mathbf{v}+\gamma\mathbf{a}} \\
  a^2=-\gamma ^6 \left[ a^2-\frac{(\mathbf{v}\times \mathbf{a })^2}{c^2} \right]
\end{gather}

\begin{align}
  \label{eq:2}
  \mathcal{L}=-\frac{mc^2}{\gamma} \\
  \mathbf{p}=\frac{d\mathcal{L}}{d\mathbf{v}}=m\gamma\mathbf{v} \\
  H = \mathbf{p}\cdot\mathbf{v}-\mathcal{L} \tag{Hamiltonian}=m\gamma c^2 =\epsilon
\end{align}
Introduciamo il quadrivettore momento:
\begin{align}
  \label{eq:3}
  p^\mu=mv^\mu=m \colvec{2}{\gamma c}{\gamma \mathbf{u}}=\colvec{2}{\frac{\epsilon}{c}}{\mathbf{p}} \\
  p^2=mc^2
\end{align}

Consideriamo ora un'onda piana, abbiamo \textbf{invarianza della fase}, poichè la fase è un conteggio di creste
\begin{align}
  \label{eq:4}
  \phi=k\cdot\mathbf{x}-\omega t=k'\cdot\mathbf{x'}-\omega' t'
\end{align}
Da qui, sostituendo $x'^\mu$ usando il boost di Lorentz, ricavo l'ultimo quadrivettore:
\begin{align}
  \label{eq:5}
  k^\mu=\colvec{2}{\frac{\omega}{c}}{\mathbf{k}}
\end{align}
Queste formule contengono l'effetto Doppler e la legge di aberrazione:
\begin{align}
  \label{eq:6}
  \omega'=\gamma\omega(1-\beta\cos\theta)
  \tan \theta'=\frac{\sin\theta}{\gamma\cos\theta-\beta}
\end{align}
\begin{align}
  \label{eq:7}
  \frac{dp^\mu}{d\tau}=F^\mu
\end{align}

\subsection{Covarianza dell'elettrodinamica}
\label{sec:covar-dell}
\begin{align}
  \label{eq:8}
  \frac{d}{d\tau} \colvec{2}{p_0}{\mathbf{p}}=\colvec{2}{\frac{q}{c} \mathbf{u}\cdot\mathbf{E}}{\frac{q}{c}( u_0\mathbf{E}+\mathbf{u}\times\mathbf{B})} %\\
%  con~u^\mu \coloneqq \colvec{2}{c\gamma}{\mathbf{v}\gamma}
\end{align}
Voglio che il membro di destra sia un quadrivett, per cui introduco:
\begin{align}
  \label{eq:1}
  J^\mu\coloneqq \colvec{2}{\rho c}{\rho \frac{dx}{dt}} \\
  \partial^\mu J_\mu=\frac{\partial \rho}{\partial t} + \nabla\cdot\mathbf{J}
\end{align}
\begin{align}
  \partial^\mu A_\mu \tag*{gauge di Lorenz} \nonumber \\
  \Box A^\mu=4\pi J^\mu  \label{eq:9}
\end{align}
Da cui:
\begin{align}
  \label{eq:10}
  F^{\mu\nu}\coloneqq \partial^\mu A^\nu - \partial^\nu A^\mu \\
  F^{\mu\nu}=
  \begin{pmatrix}
    0 & -E_x & -E_y & -E_z \\
    E_x & 0 & -B_z & B_y \\
    E_y & B_z & 0 & -B_x \\
    E_z & -B_y & B_x & 0 \\
  \end{pmatrix} \\
  F^{\mu\nu}=(\mathbf{E}, \mathbf{B}) \\
  F_{\mu \nu}=(-\mathbf{E}, \mathbf{B}) \\
  F^{*\mu \nu}=(\mathbf{B}, \mathbf{E}) \\
  F^*_{\mu \nu}=(-\mathbf{B}, -\mathbf{E})
\end{align}
Riscriviamo le eq. di Maxwell
\begin{align}
  \label{eq:12}
  \partial_\mu F^{\mu \nu}=\frac{4\pi}{c}J^\nu \\
  \partial_\mu F^{*\mu \nu}=0 \\
  \partial^\mu F^{\nu\rho}+\partial^\rho F^{\mu\nu}+\partial^\nu F^{\rho\mu}=0 \tag{forma alternativa per la seconda}
\end{align}
Posso riscrivere le eq. del moto in forma covariante
\begin{align}
  \label{eq:13}
  \frac{dp^\mu}{d\tau}=m\frac{du^\mu}{d\tau}=\frac{q}{c}F^{\mu\nu}u_\nu
\end{align}

\subsection{Leggi di trasformazione dei campi}
\label{sec:leggi-di-trasf}

\begin{align}
  \label{eq:14}
  stranote
\end{align}
Vediamo alcuni invarianti
\begin{align}
  \label{eq:15}
  \mathbf{E}^2-\mathbf{B}^2=cost \\
  \mathbf{E}\cdot\mathbf{B}=cost
\end{align}

\subsection{Lagrangiana e Hamiltoniana di particella}
\label{sec:lagr-e-hamilt}
Un po' di formule a caso
\begin{align}
  \label{eq:16}
  \mathcal{L}_{free}=-\frac{mc^2}{\gamma} \\
  \mathcal{L}\gamma=cost \\
  \frac{d}{dt}\frac{d\mathcal{L}}{d\mathbf{v}}=\frac{d\mathcal{L}}{d\mathbf{x}} \\
  \frac{d\mathcal{L}_{free}}{d\mathbf{x}}=0 \\
\end{align}

\subsection{Soluzione all'eq. delle onde in forma covariante}
\label{sec:soluz-alleq-delle}
Risolviamo l'equazione \ref{eq:9} a pagina \pageref{eq:9}, supponendo $J^\mu=J^\mu(x)$, utilizzando una funzione di Green:
\begin{align}
  \label{eq:18}
  \Box_x D(x-x')=\delta^{(4)}(x-x') \\
  z\coloneqq x-x'
\end{align}
Passando ad uno spazio di Fourier si ha
\begin{align}
  \label{eq:19}
  D(k)=\frac{1}{k\cdot k} \\
  D(z)=-\frac{1}{(2\pi)^4}\int dk D(k) e^{-ik\cdot x}
\end{align}
Risolvendo, si hanno due soluzioni:
\begin{align}
  \label{eq:20}
  D_{ritardata}=\frac{1}{2\pi}\Theta(x_0-x'_0)\delta[(x-x')^2] \\
  D_{anticipata}=\frac{1}{2\pi}\Theta(x'_0-x_0)\delta[(x-x')^2] 
\end{align}

\section{Moving charges}
\label{sec:mov-charges}
Un po' di notazione

\vspace{2mm}
\begin{centering}
\begin{tabular}[l]{c c}
  $x^\mu$ & osservatore \\
  $r^\mu$ & carica in moto \\
  $R$   & distanza fra osservatore e carica \\
  $\mathbf{\hat{n}}$ & versore dalla carica all'osservatore \\
\end{tabular}
\end{centering}
\vspace{5mm}

Posso scrivere il quadrivettore delle sorgenti per una carica in moto come:
\begin{align}
  \label{eq:21}
  J^\mu=qc \int d\tau u^\mu(\tau)\delta^{(4)}(x-r(\tau)) \\
  u^\mu\coloneqq \colvec {2}{\gamma c}{\gamma \mathbf{v}} \qquad r(t)\coloneqq \colvec {2}{ct}{r(t)}
\end{align}

\subsection{Lienerd-Wichert}
\label{sec:lienerd-wichert}
Partiamo trovando i potenziali
\begin{align}
  \label{eq:17}
  A^\mu(x)=\frac{4\pi}{c}\int d^4x'D_r(x-x')J^\mu(x')
\end{align}
Sostituendo l'eq \ref{eq:21} a pagina \pageref{eq:21}, si ottiene
\begin{align}
  \label{eq:22}
  A^\mu(x)=2q \int d\tau u^\mu(\tau)\Theta(x_0-r_0(\tau))\delta([x-r(\tau)]^2)
\end{align}
Considering the properties of the delta, and that  $\delta([x-r(\tau)])$ implies that only the points on the trajectory that lie on the backward light cone starting from $x^\mu$ can contribute to the potential (and also that $x_0-r_0(\tau_0)=\left| \mathbf{x}-\mathbf{r}(\tau) \right|$), we have:
\begin{gather}
  \label{eq:23}
  \begin{aligned}[t]
    A^\mu(x)=\frac{qu^\mu(\tau)}{u^\nu(\tau)(x-r(\tau))_{\nu}}\bigg\rvert_{\tau=\tau_0} \\
  \end{aligned} \\ 
  \begin{split}
    \Phi(\mathbf{x},t)=\frac{q}{(1-\boldsymbol{\beta}\cdot\mathbf{n})R}\bigg\rvert_{\tau=\tau_0} \\
    A(\mathbf{x},t)=\frac{q\boldsymbol{\beta}}{(1-\boldsymbol{\beta}\cdot\mathbf{n})R}\bigg\rvert_{\tau=\tau_0} \\  
  \end{split} \\
  \begin{aligned}[t]
    \mathrm{con}~ \tau_0~ \mathrm{definito~da} ~(x-r(\tau_0))^2=0
  \end{aligned}  
\end{gather}

Tramite derivazione di \ref{eq:22}, si trova il tensore del campo EM
\begin{align}
  \label{eq:24}
  F^{\mu\nu}=\frac{e}{u\cdot(x-r)}\frac{d}{d\tau}\left[ \frac{(x-r)^\mu u^\nu - (x-r)^\nu u^\nu}{u\cdot(x-r)}   \right]   \Biggl\rvert_{\tau=\tau_0}
\end{align}

E di conseguenza i campi
\begin{gather}
  \label{eq:25}
  \mathbf{E}=q \frac{\mathbf{n}-\mathbf{\beta}}{\gamma^2(1-\mathbf{n}\cdot\mathbf{\beta})^3R^2} \biggl\rvert_{\tau=\tau_0}  +\frac{q}{c}\frac{\mathbf{n}\times(\mathbf{n}-\mathbf{\beta})\times\dot{\mathbf{\beta}}}{(1-\mathbf{n}\cdot\mathbf{\beta})^3 R}\Biggl\rvert_{\tau=\tau_0} \\
  \mathbf{E}=\mathrm{c.~velocita'}(\propto \frac{1}{r^2}) + \mathrm{c.~accelerazione}(\propto \frac{1}{r}) \notag \\ 
  \mathbf{B}=[\mathbf{n}]_{rit}\times\mathbf{E} 
\end{gather}

Sappiamo che
\begin{equation}
  \label{eq:27}
  \frac{dP}{d\Omega}=R^2~ \mathbf{S}\cdot\mathbf{n}
\end{equation}
Considerando solo il campo di radiazione, e mettendoci nel caso non relativistico, otteniamo le \emph{formule di Larmor} non relativistica
\begin{align}
  \label{eq:26}
  \frac{dP}{d\Omega}=\frac{q^2}{4\pi c^3}\dot{v}^2\sin^2\theta \\
  P=\frac{2}{3}\frac{q^2}{c^2}\left| \dot{v}\right|^2
\end{align}
E le equivalenti relativistiche
\begin{align}
  \label{eq:28}
  P=\frac{2}{3}\frac{q^2}{c^2} \frac{dp_\mu}{d\tau} \frac{dp^\mu}{d\tau}=\frac{2}{3}\frac{q^2}{c}\gamma^6[\dot{\beta}^2-(\beta\times\dot{\beta})^2] \\
  \frac{dP}{d\Omega}(t')=\frac{q^2}{4\pi c}\frac{\left|\mathbf{n}\times(\mathbf{n}-\mathbf{\beta}\times\dot{\mathbf{\beta}})   \right|^2}{(1-\mathbf{n}\cdot\mathbf{\beta})^5}
\end{align}





\section{Notazione}
\label{sec:notazione}
\begin{align}
  \label{eq:11}
  X_{\mu} \tag*{covariante} \\
  X^{\mu} \tag*{controvariante} 
\end{align}


\section{M.U.}
\begin{equation}
	1 ~\mathrm{eV} \approx 1.6 \cdot 10^{-19} \mathrm{J}
\end{equation}

\subsection{Gaussian CGS}
\label{sec:gaussian-cgs}
We set the unit for $q$ such that
\begin{align}
  \label{eq:29}
  F=\frac{q_1 q_2}{d^2}
\end{align}
Thus the unit of charge is called \emph{esu}, or \emph{statcoulomb}
\begin{align}
  \label{eq:30}
  esu=\sqrt{\si{\dyne\cdot\cm^2} }=\si{\sqrt{\g\cdot\cm^3}\per\s}
\end{align}
Which results in
\begin{align}
  \label{eq:31}
  \frac{F}{L}=\frac{2}{c^2}\frac{I_1 I_2}{d}
\end{align}
We can convert a charge $q$ from CGS to SI and viceversa by noting that the Coulomb force is the same in every system (for a more thourough explanation, see \url{http://www.rpi.edu/dept/phys/Courses/PHYS4210/S10/NotesOnUnits.pdf})
\begin{align}
  \label{eq:32}
  q_{C}=\frac{q_{esu}}{10\cdot c_{SI}}
\end{align}
Where $q_C$ and $q_{esu}$ are the "number" of the respective measure unit contained in the charge $q$ (that wasn't veryclear, take a look at the link before)


\end{document}
%%% Local Variables:
%%% mode: latex
%%% TeX-master: t
%%% End:
